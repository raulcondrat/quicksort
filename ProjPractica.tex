\documentclass{article}
\usepackage[english,romanian]{babel}
\title{{\sc Raport asupra practicii: 21.06-01.07.2016}}
\author{Raul-Andrei Condrat}
\date{}
\begin{document}
\maketitle

\newpage
\tableofcontents

\newpage

\section{Introducere}

In urmatoarele capitole va fi prezentat raportul asupra practicii efectuat zilnic intre datele 26.06-07.07.2017

Ca problema de lucru am ales sa abordez algoritmul de sortare rapida QuickSort.

Quick Sort este unul dintre cei mai rapizi si mai utilizati algoritmi de sortare pana in acest moment, bazandu-se pe tehnica  "Divide et Impera".

Sortarea rapida se realizeaza �ntr-o functie care, aplicata unui sir de numere, pozitioneaza un element din sir, pivot, pe pozitia finala, pe care se va afla acesta in sirul sortat, si deplaseaza elementele din sir mai mici decat acesta in fata sa, iar pe cele mai mari dup? el. Procedeul se reia pentru fiecare din cele doua subsiruri care au ramas neordonate.
\newpage

\section{Actvitati planificate}
\begin{enumerate}
	\item 
	Luni, 26.06.2017
	
	Aducerea la cunostinta a obiectivelor si cerintelor practicii de productie
	\item
	Marti, 27.06.2017
	
	Configurarea sistemelor software pe calculatoare.
	\item
	Miercuri, 28.06.2017
	
	Studierea modului de lucru cu Git. Interfetele grafice de lucru cu Git (SmartGit)
	\item
	Joi, 29.06.2016
	
	Studierea si practicarea LaTeX
	
	\item
	Vineri, 30.06.2017
	
	Initierea algoritmului de sortare rapida
	
	\item
	Luni, 03.07.2017
	
	Lucru asupra lucrarii
	
	\item
	Marti, 04.07.2017
	
	Lucru asupra lucrarii
	
	\item
	Miercuri, 05.07.2017
	
	Prezentarea lucrarilor
	
	\item
	Joi, 06.07.2017
	
	Prezentarea lucrarilor
	
	\item
	Vineri, 07.07.2017
	
	Notarea finala a activitatii
\end{enumerate}
\newpage

\section{26.06.2017}
Studierea obiectivelor si cerintelor fata de practica de productie. Clarificarea situatiilor incerte.
\newpage

\section{27.06.2017}
Am studiat modul de lucru cu Git si interfata grafica de lucru cu Git.

Am creat repositor-ul cu numele quicksort care poate fi gasit la adresa https://github.com/raulcondrat/quicksort 
\newpage

\section{28.06.2017}
Am studiat si am practicat Latex.
\newpage

\section{29.06.2017}
Am initiat o lucrare scrisa in Latex.
\newpage

\section{30.06.2017}
Am continuat lucrul asupra algoritmului de sortare rapida QuickSort.

\newpage

\section{03.07.2017}
Am continuat lucrul asupra temei si am terminat.
\newpage

\section{04.07.2017}
Prezentarea proiectului.
\newpage

\section{05.07.2017}
Imbunatatirea proiectului.
\newpage

\section{06.07.2017}
Prezentarea proiectului.
\newpage

\section{07.07.2017}
Notarea finala a activitatii.
\newpage

\section{Concluzii}
Am invatat sa lucrez cu Latex, Git si BitBucket.

Mi-am dat seama ca documentele PDF sunt atat de populare datorita complexitatii algoritmului LaTeX spre deosebire de alte documente similare.

GIT mi se pare un sistem foarte interesant pe care cu siguranta il voi folosi foarte mult in viitor avand in vedere facilitatile pe care le-am descoperit.
\newpage

\addcontentsline{toc}{section}{14 Bibliografie}
\begin{thebibliography}{9}
\bibitem{thebookwithouttitle} 
J. De
\textit{The Book without Title}. 
Dummy Publisher, 2100.
 
\bibitem{simplelaex} 
G.H. J.
\textit{A Simplified Introductin to LaTeX}. 
2100.

\end{thebibliography}

\end{document}